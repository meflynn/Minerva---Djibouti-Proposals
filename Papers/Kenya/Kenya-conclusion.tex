\section*{Conclusion}

The results indicate that interactions with members of the United States or China act as a powerful force in shaping people's attitudes. Contact is a blunt instrument representing a range of interactions, from passing someone in the grocery store and sharing extensive time with someone due to work or social situations to long-term intimate relationships. However, despite this range, it is increasingly evident that foreign nations, regardless of their official status, are essential diplomats on behalf of their countries. This is true whether they are service member, tourists, overseas employees, or government officials. 

Our research indicates that there is more to explore in understanding how the United States and China export influence abroad and how that connects to national and international levels of foreign policy. The Domain of Competitive Consent argues that individual attitudes are increasingly becoming important in determining security policy; understanding the microfoundations of alliances, military access, and basing means moving from elite decision-making models to understanding how the public views the actors in Great Power Competition.

The exploratory results in this paper demand further exploration. Experimental research can provide added depth to understanding what kind of contact matters and how. While the observational work we do in this study indicates that it matters, retrospective questions cannot uncover the causal mechanisms that lead to attitude shifts. Setting up a non-hypothetical experiment, however, requires extensive intervention and follow-up that research on attitudinal shifts has only recently begun to uncover \cite{broockman2016,kalla2020}

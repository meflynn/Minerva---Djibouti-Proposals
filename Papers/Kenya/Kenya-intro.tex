The opening of China's People's Liberation Army base in Djibouti in 2017, its first permanent military installation abroad, was a key moment for Chinese foreign policy. It launched discussion and speculation about whether China's growth in power and influence would lead it to seek a foreign military base network to rival that of the United States \cite{vertin2020}. Yet, in the seven years since, we have not seen as strong of an effort from China to continue to build military installations abroad. Ream Naval Base in Cambodia was renovated with Chinese funds, and Chinese military vessels have docked there since 2023 and maintained access \cite{gan2023}. Beyond that, while there has been talk of China's interest in locations such as the UAE or the Solomon Islands, there has yet to be a sharp increase in Chinese military installations.

Does this mean that China does not have global military ambitions? Is it scaling back on its quest for global political influence? We argue that this is not the case but rather that China is following a pattern of influence that is different from that of the United States and other former colonial powers such as Britain and France. By forcing China into the U.S. model of foreign military influence and expecting it to set up military bases under the American model, we are missing observations and lacking a fuller understanding of Chinese foreign military policy. 

The ability to project military power abroad is a central means by which states exert influence in international politics, allowing them to defeat adversaries and reassure allies across large distances \cite{levy2010,markowitz2013,blankenship2022}. To project power, states need access to other countries' territory, often in foreign military bases, which allow states to control territory, forward deploy personnel, and resupply their forces \cite{Harkavy1989,posen2003}. Basing access, however, is often precarious. In the current international system, states typically rely on the consent of sovereign host states to build and maintain their foreign military presence. This strategy contrasts with earlier periods when states secured bases through force, coercion, and formal empire \cite{schmidt2020}. Because host states can grant or deny access, they are subject to pressure as great power rivals like the United States and China increasingly compete for military access and influence worldwide.

The systematic study of overseas military basing and how it affects and is affected by host populations is still new. While military basing is a centuries-old practice, methods of securing military access have evolved. Traditionally, the ability to deploy troops in other states' territory primarily arose from conquest and colonialism. After World War II and the period of decolonization, alliances, and regime change provided a path for the United States and the Soviet Union to have long-term military access to (or control over) others' territories. While the collapse of the Soviet Union led to the withdrawal of most Russian bases, the United States expanded its network to cover most of the globe.

In several ways, the emerging era of great power basing is distinctive from either the Cold War or post-Cold War periods. First, unlike the post-Cold War period, the United States faces geopolitical competition from near-peer great powers like China. As a result, the challenges to U.S. bases no longer stem solely from internal forces—concerns among the population about crime, pollution, and infringements on sovereignty—but also have an essential external component\cite{allen2023}.

Second, unlike the Cold War, China's challenge to U.S. bases is often more indirect and asymmetric than that of the Soviet Union. The United States and the Soviet Union competed for bases directly but rarely had bases in the same countries \cite{Nieman2020}. China, by contrast, has pursued a lighter-footprint military presence model and has often sought access and influence in countries that could or already do host a U.S. base. For example, China placed its permanent military base in a country that already hosted Africa's largest U.S. base. Similarly, much of China's effort has focused on establishing military access to other states' existing military or commercial installations, such as dual-use commercial ports \cite{kardon2022}. China has thus followed a different power projection model that better fits with a strategy of avoiding confrontation and building political, economic, and military influence \cite{Doshi2021}. The aforementioned example of gaining access to Cambodia's Ream Naval base fits this model \cite{gan2023}.


In this project, we take the case of Kenya, which allows us to observe the American and Chinese models for military access and influence side by side. The United States maintains a military presence at the Manda Bay Airfield in southeastern Kenya, where the Kenyan Defense Forces operate, but also hosts a U.S. military presence of approximately 50 troops \cite{allen2022}. China's influence on Kenya is mainly expressed through large amounts of foreign direct investment in infrastructure through the Belt and Road initiative and sovereign debt incurred from China \cite{lesutis2021}. Although China has not yet sought (to our knowledge) a permanent military base in Kenya, it has sought subtler forms of access in line with its prevailing model, making it a more typical case for Chinese pursuit of access. A 2023 U.S. Department of Defense report notes that China's People's Liberation Army Strategic Support Force ``operates tracking, telemetry, and command stations'' in Kenya \cite{DOD2023}. 

%The United States and China have increasingly sought to influence African nations through economic, military, and diplomatic channels to gain military access to countries. Such projects include China's Belt-and-Road Initiative, the United States constructing limited footprint bases like the drone base in Niger, and both countries maintaining a military presence in Djibouti. As part of an exploratory analysis to see how influence campaigns in Africa affect the perceptions of host-state civilians, we deployed a survey in Kenya. Kenya represents a case where the country has stronger military ties to the United States while also experiencing active influence efforts by China. Given that military access increasingly requires the consent of the public and not just elites, the perception of civilian actors can influence Kenya's military alignment in the long run. 


A recent U.S. Institute of Peace (USIP) report notes that many African states are wary of military presence within their territories. Though a military presence or base can be profitable for leaders (see Djibouti as an extreme example), ``military bases can also be a political liability for African governments that host them'' \cite{usip2024}. This opinion is not a new sentiment. As noted in the report, a 2016 African Union report ``noted with deep concern the existence of foreign military bases and establishment of new ones in some African countries'' and ``stressed the need for Member States to be always circumspect whenever they enter into agreements that would lead to the establishment of foreign military bases in their countries'' \cite{AU2016}. Thus, if China and the United States want to expand or maintain their military presence in Africa, they will have to convince elites and members of the population to accept this presence, whatever shape it may take. This need for persuasion will be particularly true in democratic or partly democratic states such as Kenya. 

This project thus compares the two influence models of the United States and China by studying Kenyans' perceptions of the United States and China generally and their support for hosting a Chinese and U.S. military presence in their country. Following previous work, we expect that both personal interactions with individuals from a major power, as well as the cultural influence that is created from that major power's presence or investment in the country (be it military or not) will lead to more positive views of the major power, as well as greater acceptance of granting the major power access to the country. Regarding the effectiveness of the Chinese vs the American approach, we remain agnostic and treat the study as exploratory in comparing Kenyan views of both countries. To test these questions, we deployed a survey to Kenya in the summer of 2023 to assess the cross-competing effects of influence campaigns by the United States and China. Despite Kenya's ties to the United States, Chinese influence campaigns in the region are prominent and effective. 







\section*{Research Design}

To test our hypotheses, in September 2023, we developed and deployed a survey instrument in Kenya. We deployed the survey in English and Swahili to 1,023 Kenyans via SMS (text message) using the survey firm GeoPoll, which has extensive experience in Kenya, to implement the survey. Respondents were over 18 years of age and recruited by Geopoll using a database of mobile phone numbers across Kenya.\footnote{While most people reported their primary language being Swahili, most respondents opted to take the survey in English} The survey was nationally representative on age and gender. GeoPoll incentivized respondents with airtime credit (about 0.50 USD, in local currency) to respond to the survey, which took approximately 15 minutes to complete, on average.\footnote{0.50 USD translated to 69 Kenyan Shillings at the time of the survey. This would roughly cost a 12 oz bottle of Coke or Pepsi. This amount met the minimum threshold necessary to motivate participation based on GeoPoll's experience operating in Kenya.} Though this limited the sample to only people who own mobile phones, mobile use is widespread enough in Kenya that this did not limit our sample in a significant way. We collected the full sample over six days.  

In the survey, we intentionally oversampled respondents from Mombasa County (50\% of respondents), as this is where the U.S. maintains its military presence. The rest of the sample is nationally representative of the Administrative 1 (ADM1) location. Oversampling Mombasa County increases the probability of respondents having had contact with the U.S. military, thus allowing us to assess whether contact with service members influenced views of the U.S. or its military presence. Though it would not give us an accurate representation of how likely an individual would be to have contact with a member of the U.S. military, in general, we note that our hypotheses are not about the frequency of contact but about the effect that contact can have on views of major power military access. 

The survey contained 28 questions. It included demographic questions of age, gender, language spoken at home, household income, and the county in which the respondent lives. These questions allow us to sample representatively and control for demographic factors that may influence views on foreign military access.

\subsection*{Dependent variable} 

Our hypotheses focus on two dependent variables: General views of a major power (either China or the U.S.) and support for hosting a major power's (China or the U.S.) military in Kenya. 

To measure the first dependent variable (general positive or negative views on a major power), the survey asked questions about the perceived influence of the major powers on Kenya (``How much economic and political influence does China/the United States have on Kenya?''), with answer options being ``A lot'', ``Some,'' ``A little,'' ``None,'' and ``Don’t know / haven’t heard enough.'' We also asked whether that major power's influence was positive or negative, with response options being ``Very negative,'' ``Somewhat negative,'' ``Neither positive nor negative,'' ``Somewhat positive,'' and ``Don't know/no opinion.'' We also asked respondents about which country they thought ``should be a model for [Kenya's] future development''. Besides China and the United States, we also 
included Britain (as a former colonial power), other African states, India, and options for ``Other'' and ``None of these,'' not to force respondents to choose only between the U.S. and China.

To measure the second dependent variable, the level of support for a major power's military presence in Kenya, we ask respondents ``In general, what is your view toward the United States/China having a military presence in Kenya?'' The options again ranged from ``Very Negative'' to ``Very Positive,'' with the option to respond ``Don’t know / Decline to answer.''\footnote{We note that before asking about the respondents' views towards the foreign military presence, we first asked them if, to their knowledge, the United States/China has a military presence in Kenya.}  

\subsection*{Independent variables} 

Our hypotheses also focus on two key independent variables (both of which we expect to influence both views of the major powers and support for their military presences). The first is interpersonal contact, and the second is cultural influence. They are both measures of soft power, but one is a direct social interaction, whereas the other measures the exposure of individuals to media such as film, TV programs, sporting event broadcasts, social media, and foreign education destinations. 

For contact, we first asked respondents if they had had ``face-to-face contact with a citizen of the United States/China in Kenya?'' If they responded yes to this question, we then asked them several follow-up questions about the nature of that contact. We first asked them about the frequency of contact, ranging from ``Daily'' to ``Once.'' The next question asked about the type of contact had, giving them a list of options for forms of contact (ranging from ``a brief everyday interaction'' to ``Dated/were romantically involved with'' a person from the major power of interest).\footnote{See appendix for the full list of response options} Respondents were allowed to select all options that applied to them and an ``Other'' option. Finally, we asked whether any of that contact was with ``a member of the U.S./Chinese military'', to determine whether any direct contact was happening between deployed military personnel and locals in Kenya. In the survey we first asked respondents this set of questions about the U.S. and China.

To ask respondents about their exposure to cultural influence, we asked the about the country of origin of ``the last film you watched in a theater,'' ``the last television program you watched,'' and ``the last athletic event you watched.'' In addition, we asked them about the social media app that they used most often, to gauge the popularity of U.S.-owned apps such as Meta, Whatsapp, or Instagram or Chinese-owned ones such as TikTok. Finally, we asked about any family members who had gone abroad to study, and which country was the most common destination.\footnote{Besides China and the U.S., the other options given were ``Britain,'' ``Ethiopia,'' ``South Africa,'' and ``India,'' as well options for ``Other'' and ``None.''}



\subsection*{Control variables} 

In addition to our independent variables, we include a series of demographic questions that could affect perceptions of China and/or the U.S. and the likelihood of having contact with foreign citizens and being exposed to foreign cultures \cite{clarke2005}. We asked about income levels, as wealthier individuals may have more cosmopolitan attitudes about foreign states, have traveled more, and are more receptive to foreign influences. We also included a variable for the highest level of schooling completed, again with an expectation that education may correlate positively with openness to foreign ideas and individuals.

In addition, because past work has found that women may be less receptive to militaristic policies than men, we also take into account the respondent's sex \cite{allen2020,allen2023}.\footnote{At the advice of the survey firm's country expert who noted this issue could be culturally sensitive, we did not include a ``non-binary'' option in this question, instead asking about sex (rather than gender) and included a ``Prefer not to say'' option.} Finally, because we expect that young people may both be more likely to hold liberal, anti-militaristic policies but also be more likely to be exposed to foreign media and influences, we asked about the respondents' age. 



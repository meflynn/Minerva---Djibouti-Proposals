\section*{Results}


The preliminary results point to some interesting dynamics. 40\% of respondents view China as having both ``A lot'' of influence and ``Somewhat positive'' or ``Very positive'' influence. Similarly, around 44\% of respondents view the United States as having ``A lot'' of influence in Kenya and view that influence as ``Somewhat positive'' or ``Very positive.'' Regarding military deployments, only 20\% of Kenya respondents believed that China has military personnel operating in Kenya, compared with 72\% of respondents who correctly responded that the United States has military personnel deployed to Kenya. Of those correctly identifying a Chinese military presence in Kenya, 63\% viewed that presence as ``Very positive'' or ``Somewhat positive.'' Similarly, of those correctly identifying a U.S. military presence in Kenya, 70\% % viewed that presence favorably. Figure 1 shows the joint distribution of respondents' answers to the questions about the presence of deployments and their evaluations of those deployments. 




These preliminary results suggest that public views of these major powers are comparable and potentially well-positioned for competitive influence campaigns. However, the United States and China also proceed from different starting points. A 2023 U.S. Department of Defense report notes that China's People's Liberation Army Strategic Support Force has limited personnel operating in Kenya. It expects China's pursuit of basing access to grow (DoD 2023). The United States has relatively high favorability levels and a long track record of basing in foreign countries. Given its relatively small military footprint to date, it remains to be seen if China can sustain such high levels of public approval as it expands the scope of its military-basing activities. Understanding the factors that shape public assessments of the costs and benefits of foreign basing will be key to understanding how this process will play out for the United States and China as they compete for access.

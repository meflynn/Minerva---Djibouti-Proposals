\section*{Theory}

Since the end of the Cold War the United States has maintained an unrivaled network of global military bases. This access to foreign military territories has allowed the United States to maintain a global sphere of influence, assure allies, and threaten intervention in response to emerging threats. However, even a global basing network does not provide unfettered access. An example of the United States being constrained by its basing hosts occurred in the lead-up to the U.S. invasion of Iraq in 2003. The United States, which maintained between 2,000–3,000 troops in Turkey, requesting the use of Turkey as a launching pad for its invasion. Unexpectedly, the Turkish parliament refused the U.S. request. The unexpected negative vote was attributed to many factors, including disunity within the ruling party, a Turkish attempt to gain more concessions from the United States, and Turkish public opinion that was overwhelmingly against the war \cite{otterman2005}. 

As this example illustrates, access is an ongoing political and bargaining process that requires consent from the granting state. Moreover, this bargaining process often encounters internal challenges from domestic audiences within a host country who oppose the foreign military presence, or the actions that may be launched from the host territory, as was the case in Turkey and as has been the case in German protests against U.S. drones being operated from Ramstein Air Base. Internal challenges can be ideologically rooted, such as opposition to infringement upon host sovereignty or pacifist sentiment, or may derive from the negative effects that bases impose on their environments, including noise, environmental pollution, traffic congestion, and crime committed by the basing power's service members \cite{kim2023}.

To overcome these potential objections, basing countries use different policy tools to win both elite and popular support for a military presence. These include using financial incentives to curry favor---such as building infrastructure for the host population and hiring local labor—as well as taking steps to ensure that the host population has positive social interactions with military personnel \cite{allen2020,blankenship2020,allen2023}. As reported by the Washington Post in April 2023, the U.S. Department of Defense believes that ``the PLA likely will use tailored approaches to address local concerns as it seeks to improve relations with amenable countries and advance its overseas basing goals'' \cite{hudson2023}. The U.S. State Department similarly noted in 2018 that ``the United States has created a strong Djiboutian constituency that favors our military presence, owing to increased local hiring and contracting with Djiboutian companies at Camp Lemonnier'' \cite{state2018}. 

Despite the importance of these tools in cultivating host support and the centrality of host support to bargaining over access, both support and its drivers remain understudied. This lack of research depth remains particularly true in the present era of competition between the United States and China. Moreover, much of China's challenge to U.S. bases is economic rather than directly military in nature. China has emerged as a massive source of global lending and foreign investment, particularly since the announcement of the Belt and Road Initiative in 2013, and many recipients of Chinese investment are U.S. base hosts, while the United States is also a major investor and uses economic power to create security influence. This type of economic influence from a rival power poses at least two challenges to a competitor's basing efforts. First, that rival can actively seek to use its economic incentives to convince policymakers or the public within a host state to block access to a rival power seeking to build or expand a new military base \cite{joyce2023}. For example, the United States has actively pressured its ally, the United Arab Emirates, to scrap a plan to allow for the installation of a Chinese military facility near Abu Dhabi \cite{hudson2023}. Second, the rival's economic engagement can crowd out the incentives the basing country offers, making the incentives seem less attractive by comparison \cite{joyce2023}.
%[Probably need to add more here. Maybe something about economic influence translating to cultural influence, which is going to be what we argue in the hypotheses]

\subsection*{Economic and Cultural Influence}
%[RJ: Nye and others generally include economic threats and payments under hard rather than soft power. Probably either modify this framing to make it about hard and soft power (e.g., Nye's "smart power", or explain how/why we are treating economic incentives as soft power here)] MA: Excluded economic threats, only included economics in the soft sense that nye does in terms of "attractiveness of the economic model'', etc. 

Soft power as a force for influence in international relations gained prominence as the Cold War was dying down \cite{nye1990}. The idea that norms, culture, and ideas can influence international relations is certainly older than the coining of soft power. However, with the transition from bipolarity to hegemony, international relations scholars found new interest in soft power. 

Soft power can work through a few different mechanisms. Notably, it is not directly coercive as hard power in that the pulls of economic models, ideas, and culture become a gravitational force for other countries and their populations. Different actors may see another country's success and wish to model their institutions after that country to emulate their successful model. Alternatively, people may adopt the ideas and beliefs of the soft power-producing country and see the world through similar eyes. Soft power works well when actors agree upon their vision for the world and the paths to achieve those goals.

The nature of soft power and the instruments that contribute to it evolve with culture and economics. Much as globalization has reshaped communication, trade, and culture, globalization is an ongoing process that also alters soft power tools. While music became a catalyst for soft power during the Cold War and remains vital in expressing cultural values, other venues continue to emerge and establish themselves \cite{nye2004}. Likewise, sports and movies are classic conduits of soft power\cite{sari2012,grix2014}. Social media created a new wave of ways in which individuals and states can access individuals directly, and the occupation of influencer arose in the 2000s.\footnote{The role of influencer is certainly historic and predates the modern incarnation. For example, movie celebrities and the rise of the fashion industry was an incarnation of the influencer role.} Scholars have noted how influencers within a country can promote the messages of foreign governments through social media channels \cite{vibber2021}. 

The role of influencing people through online media is not unique to just personalities but may concern the platform those personalities occupy. Initially, much of the dominant social media came from Western countries through Myspace, Facebook, YouTube, Reddit, and Instagram, but Chinese companies have competed in that space with Weibo and TikTok. In recent years, there has been growing concern about which countries control which social media platforms, what those companies and governments do with the data from the platform, and how that can influence society, culture, and politics.  

The United States, Russia, and China have engaged in extensive campaigns of economic and cultural influence throughout the globe. The attractiveness of an economic model's success is an important factor in accumulating soft power \cite{mcclory2010}.  Influence sources for these competing actors exist with official policy, such as the promise of successful development through the Belt-and-Road initiative, social media, and interpersonal contact between foreign nationals and civilians. Soft power can stem from Chinese and American nationals in other countries for personal, business, or official state capacities. Deployed military members overseas are a source of soft power that builds support for their missions through their official activities and everyday activities as private individuals \cite{atkinson2014,allen2023}. Patronage of local establishments, raising a family in a host community, or attending local events can normalize and encourage the support of a foreign military presence. 

In all these ways, soft power is vital in securing military access and building support from the ground up for a nation becoming aligned with another government.


\subsection*{The Politics of Influence and Military Access} 

Early scholarship on foreign military basing focused on understanding how great powers acquire, use, and compete for bases. Perhaps most notably, Robert Harkavy's \citeyear{Harkavy1982,Harkavy1989,Harrison2000} work offered sweeping accounts of great power bases over eight centuries, with particular emphasis on how the United States and Soviet Union attempted to acquire bases and deny them to each other using various economic, political, and military tools. Other scholars focus primarily on the United States, attempting to chronicle the scope and purpose of the vast U.S. basing network during and after the Cold War \cite{vine2015,moore2016,sandars2000}. This body of scholarship, however, is largely descriptive and almost exclusively focused on government-to-government interactions.

More recently, scholars have opened the black box of basing relationships to explore how domestic politics in host countries can shape the political viability of overseas bases and how basing countries, in turn, can adapt. \citeasnoun{calder2008}, \citeasnoun{Cooley2008}, and \citeasnoun{yeo2011} all shed light on how domestic anti-base movements can pressure host governments to evict foreign militaries, particularly during periods of democratic transition. Building on this work, \citeasnoun{allen2023} explore the micro-foundations of domestic support for foreign bases using surveys across fourteen countries, with findings suggesting that positive economic and social interactions between U.S. personnel and the host population can build support for the U.S. military presence.

These two strains of literature have mainly remained separate, leaving a gap in our understanding of how great power competition can shape the foundations of domestic support for hosting foreign bases. This competition can be directly military, as in the case of U.S.-Soviet competition for bases during the Cold War \cite{Nieman2020,Harkavy1982}. It can also be broader, with rivals seeking political, economic, and cultural influence across the same countries. For example, China has primarily sought influence with economic tools, perhaps most notably through its Belt and Road Initiative, which has financed some \$500 billion in infrastructure globally since 2008. China has relied on access to infrastructure like ports through the ownership rights of state-owned enterprises to project power, as in Cambodia \cite{kardon2022,kardon2022pier}. Additionally, China acquired the rights to its first foreign military base in Djibouti in 2015, and since then has sought base rights in countries across Asia, Africa, and even the Americas \cite{hudson2023,strobel2023}. Leaks of Department of Defense documents, reported by the Washington Post in April 2023, revealed U.S. military estimates that ''the PLA seeks to establish at least 5 overseas bases and 10 logistic support sites by 2030 to fulfill Beijing's national security objectives, including protecting its economic interests abroad''\cite{hudson2023}. Even where China does not seek bases, its economic footprint poses problems for the United States, as China can use its influence and economic leverage to deny U.S. access. In Kenya, for example, U.S. officials have indicated their alarm at the country's willingness to hire a Chinese construction firm to complete upgrades to a joint Kenya-U.S. counter-terrorism base unless the United States pays for the upgrades itself, fearing that the Kenyans could leverage geopolitical ties with China for economic gain \cite{philips2023}.

This project attempts to fill this gap by exploring how major powers can derive influence in foreign countries through military, social, economic, and cultural contact with local populations. It argues that whether this is an intentional strategy or not, this influence can facilitate the acquisition and maintenance of military access to these states. Importantly, we study this question considering that more than one major power may be competing for access to these third-party states. One major power's influence campaigns may limit its rival's ability to gain military access to a third-party state. We will thus explore how major powers' influence attempts can be undermined (or not) by those of rival major powers. We will shed light on additional mechanisms through which great power competition can shape foreign military deployments beyond government-to-government interactions. 

From the military basing side, much of the literature has focused on great power competition in U.S.-Soviet Cold War relations. The host countries studied have generally been in regions like Europe and East Asia. The literature focused on major powers' quest for influence in Africa has tended to study it in terms of influence by former colonial powers, or with a greater focus on purely economic investment and aid. Yet there is a lack of work on the competition for military access to African states, particularly in democratic ones like Kenya. In the current environment, competition for bases and military access occurs in a world characterized by strong norms of sovereignty, in which access must be granted consensually and in which many hosts are either democratic or could democratize \cite{Cooley2008,schmidt2020}. Therefore, understanding the mechanisms that govern the consent of domestic populations toward foreign military basing and deployments is crucial for comprehending the conditions under which great powers can project power abroad. 

What instruments are likely to build consent among domestic populations? A well-studied source of power in international relations is the ability to change other actors' incentives to make it rational for them to comply with an actor's preferences \cite{dahl1961}. Aside from the threat or use of force—which has over time become a less common means of securing foreign bases—states can use positive inducements to structure other actors' incentives \cite{schmidt2020,Lake1996}. Indeed, the literature suggests that states often use tools of economic statecraft like foreign aid to buy foreign policy influence and secure access to bases \cite{carter2015,alexander2019,blankenship2020,joyce2023}. However, there is some evidence that this effect may vary across states. A recent study of U.S. and Chinese aid to 38 different African countries found a link between U.S. aid and positive views of the United States. However, Chinese aid did not affect public support or actively reduce it \cite{blair2022}. This divergence suggests that major powers who compete for public approval using similar tools may not achieve similar effects, and highlights the need for research to study how such interventions may uniquely affect civilians in host countries.

Beyond structuring other actors' incentives, states can also attempt to elicit cooperation through soft power. Overseas military deployments can be a source of soft power \cite{atkinson2014}. First, the most obvious way military deployments can encourage soft power is through humanitarian missions where service members assist with health care or disaster relief. These acts build support for the basing country as it is clear that the assistance comes from the base power \cite{flynn2019}. Second, service members integrated into overseas communities can build soft power, although this is harder to observe. Routine daily behavior by service members on and off base creates potential points of interaction that can build support for a basing country's mission in a host country. Research finds that interactions with service members can reduce stereotypes, build goodwill, and humanize a deployed force. Contact alone can produce positive assessments of a foreign-deployed army \cite{allen2023}. States with an active, non-isolated presence can actively build support for their presence with local populations. As such, we expect interactions with U.S. and Chinese individuals will garner support for each country and its military within Kenya. This logic creates our first set of hypotheses:

\begin{subhyp}
	
	\begin{hyp}
		Interpersonal contact with U.S. individuals correlates with more positive views of the United States.
	\end{hyp}
	
	\begin{hyp}
		Interpersonal contact with Chinese individuals correlates with more positive views of China.
	\end{hyp}
	
\end{subhyp}

\begin{subhyp}
	
	\begin{hyp}
		Interpersonal contact with U.S. individuals correlates with more support for hosting the U.S. military.
	\end{hyp}
	
	\begin{hyp}
		Interpersonal contact with Chinese individuals correlates with more support for hosting the Chinese military.
	\end{hyp}
	
\end{subhyp}

Moreover, given our discussion about the role of cultural influence through sporting events, entertainment, and other advertising vehicles, we expect sources of cultural influence to correlate wit positive views of each country and its military in our second set of hypotheses.


\begin{subhyp}
	
	\begin{hyp}
		More exposure to U.S. cultural influence correlates with more positive views of the United States.
	\end{hyp}
	
	\begin{hyp}
		More exposure to Chinese cultural influence correlates with more positive views of China.
	\end{hyp}
	
\end{subhyp}

\begin{subhyp}
	
	\begin{hyp}
		More exposure to U.S. cultural influence correlates with more support for hosting the U.S. military.
	\end{hyp}
	
	\begin{hyp}
		More exposure to Chinese cultural influence correlates with more support for hosting the Chinese military.
	\end{hyp}
	
\end{subhyp}

Additionally, existing scholarship indicates rival providers can undermine states' influence attempts. The literature on foreign aid and economic statecraft, for example, suggests that states and international organizations like the World Bank are less able to make their assistance conditional on policy concessions when recipients have alternative sources of aid and financing \cite{dunning2004,bdm2016,woods2008,kastner2021,watkins2022}. %Do we want hypotheses for these?


However, the literature leaves gaps in our understanding of great power inducements and influence. For one, tools of influence—such as foreign aid, military contact, and military training—are typically studied independently rather than comparatively. Research on influence and policy concessions also tends to ignore military bases and focuses on government-to-government interactions rather than government-to-public interactions, while work on influence and public opinion tends not to focus on public support for policy concessions and tends to ignore the role of foreign competition. Indeed, while the U.S. Department of Defense worries that the PRC targets countries for future military installations, we know little about the ``host nation receptivity'' to these intentions \cite{hudson2023}. 

Thus, given existing literature on the role of international competition in undermining states' influence attempts \cite{joyce2023}, we might expect that exposure to contact with U.S. and Chinese individuals, economic footprints, and culture will negatively correlate with support for hosting the other's military presence.

\begin{subhyp}
	
	\begin{hyp}
		More exposure to U.S. individuals and influence correlates with less support for hosting the Chinese military.
	\end{hyp}
	
	\begin{hyp}
		More exposure to Chinese individuals and influence correlates with less support for hosting the U.S. military.
	\end{hyp}
	
\end{subhyp}


As we describe in more detail below, we test these hypotheses using a survey instrument in which we ask respondents about their contact with U.S. and Chinese citizens and their exposure to the culture of one or both of these major powers. We then ask respondents about their level of support for those countries' bases.







